\chapter{Exploring Visual Features}
\label{chapter:VisFeatChapter}
%%===========================================================================%%
\section{Image Feature Extraction}
\fixme{Upto 4 pages: Mostly from Paper and Report, except the visualizations}
\label{sec:ImageFeat}

SHOW VISUALIZATIONS IF POSSIBLE
    --> Feature vector TSNE
    --> Non obvious information in object detectors, For eg. difference between
    using binary vs real valued detector scores
%%----------------------------------%%
\subsection{Convolutional Neural Networks}
%%----------------------------------%%
\subsection{Object Detectors}
%%----------------------------------%%
\subsection{Spatial Map Encodings}
%%----------------------------------%%

%%===========================================================================%%
\section{Video Feature Extraction}
\fixme{Upto 3 pages: mostly fresh writing}
\label{sec:VideoFeat}

SHOW VISUALIZATIONS of Trajectories
%%----------------------------------%%
\subsection{Key-frame and multi-frame features}
ALSO COVER POOLING HERE
RANK POOLING???
%%----------------------------------%%
\subsection{Trajectory Features}
%%----------------------------------%%
\subsection{3D Convolutional Network}
%%----------------------------------%%
\subsection{Feature Encoding Network}

% Comment: If your sentence ends in a capital letter, like here, you should
% write \@ before the period; otherwise LaTeX will assume that this is not
% really an end of the sentence and will not put a large enough space after the
% period. That is, LaTeX assumes that you are (for example), enumerating using
% capital roman numerals, like I. do something, II. do something else. In this
% case, the periods do not end the sentence.

% Similarly, if you do need a normal space after a period (instead of
% the longer sentence separator), use \  (backslash and space) after the
% period. Like so: a.\ first item, b.\ second item.

