\chapter{Enhancing Language Model}
\label{chapter:langModel}
%%===========================================================================%%
In this chapter we will look at several extensions to our baseline language
model.
%%
The first extension is to add an additional feature input channel to the
language model aimed at helping us effectively utilize the multiple new image
features proposed in chapter~\ref{chapter:VisFeatChapter}.
%%
We then discuss our adaptation of the residual connections proposed for CNNs
recently to the LSTM language models.
%%
This allows us to effectively train deeper LSTM language models.
%%
Next we look at utilizing a hierarchical factorized decoder at the language
model output, in a bid to produce richer captions.
%%
Finally we also present a few ensembling techniques we use to combine multiple
language models.

\section{Additional Feature Inputs to Language Model}
%%
The language model we use should be capable of utilizing different kinds of
visual features presented in chapter~\ref{chapter:VisFeatChapter}
simultaneously.
%%
But the baseline language model, presented in chapter~\ref{chapter:baseline} has
only one input channel, shared between word vectors and image features.
%%
Hence, here the visual features are input to the network only in the zeroth
round of iteration.
%%
We refer to this input channel as the \emph{init} input to the LSTM network.
%%
This technique was also proposed in~\cite{Vinyals_2015_CVPR} as a solution to
prevent overfitting. 
%%
Therefore, the only way the baseline language model can utilize multiple visual
features is if they are fused into a single vector before presenting to the
language model.
%%
But in our experiments, we found that performing simple feature fusion, like
concatenating the two feature vectors, and using it as as input in the baseline
language model leads to inferior performance.
%%

To address this issue we introduce another data input channel to our
language model and make the visual features available through the whole
inference process.
%%
This requires adding a new input path to the LSTM cell which we refer to as the
\emph{persist} input.
%%
This is illustrated in figure~/ref{fig:proplstmlang}.
%%
Note that the persist input plays the same role as $x(t)$ in
equations~(\ref{eqn:lstmstrt})--(\ref{eqn:lstmend}), but it has its own set of
input weights.
%%
For brevity, we won't repeat them here.
%%
Accordingly, the proposed framework computes the distribution:
%%
\begin{equation}
p(w_t | w_{t-1},\cdots,w_0, I, P) = \text{softmax}(W_d y(t)) \;,
\end{equation}
%%
\noindent where $I$ and $P$ represent the \emph{init} and \emph{persist}
features respectively.
%%
Training is again done by minimizing the negative log likelihood cost function
as in the baseline model
%%
\begin{align}
  %\label{eqCost}
  \mathcal{L}(w_{1\cdots L} | I,P) = -\sum_{t=1}^n \log(p(w_t|w_{t-1},I,P)) \; .
\end{align}

%%
Having separate input matrices for \emph{init} and \emph{persist} features
allows the model to learn different functions from the word embeddings and the
visual features and in-turn makes the language model more powerful.
%%
We can also now input different visual features in the \emph{init} and
\emph{persist} paths thereby allowing the model to learn simultaneously from two
different sources.

\begin{figure*}[t]
\begin{center}
  \includegraphics[width=0.7\linewidth]{images/MultilayerResidualLSTM.pdf}
\end{center}
\vspace*{-4mm}
\caption{The proposed language model architecture. The dashed blue lines
        highlight the changes we have proposed over the baseline. 
        Here we show a two-layer LSTM with residual connections is
        shown.}
\label{fig:proplstmlang}
\end{figure*}

%%===========================================================================%%
\section{Deeper Models Using Residual Connections }
%% ---------------------------------------------------------------------------
Another extension we experiment with to improve the performance of our language
model is to add depth to the LSTM network used in it.
%%
Here, only the first LSTM layer receives the feature inputs directly and the
higher LSTM layers take their input $x(t)$ from the previous layer in the
network.
%%
The recurrent connections from a network's outputs back to its inputs exist only
within an LSTM layer and not across the layers.
%%
Softmax is applied at the output of the last LSTM layer only.
%%

Recently residual connections was proposed in~\cite{He2015} to be used in CNNs.
%%
It consists of adding a fixed identity connection from output of a lower layer,
$f_1(x)$, to the output the layer above it, $f_2(x)$.
%%
This alters the output after two layers from $f_2(f_1(x))$ to $f_1(x)+f_2(f_1(x))$.
%%
Now the second layer only needs to learn only a residual function to push the
output of the first layer towards the desired output.
%%
If the first layer is already performing well, this makes the task of second
layer much easier.
%%
This seemingly simple technique allowed to train much deeper CNNs~(almost 3
times of the GoogLeNet) and achieve state-of-the-art performance on ImageNet
dataset in image classification task.

Inspired by this success, we adapt this method to our LSTM language model and
add residual connections between LSTM layers. 
%%
These residual connections, shown in Figure~\ref{fig:proplstmlang}, improve the
training convergence speed greatly.
%%
We have also found out that the use of residual connections produces
significantly lower values for both the training cost function and the model
perplexity, as will be presented in Section~\ref{chapter:results}.

%%===========================================================================%%
\section{Hierarchical Decoder}
\label{sec:ClassFact}
Analyzing the baseline language model outputs for errors, one category of
mistakes we often see are where the model misses the fine grain
classification between two closely related objects.
%%
For example we have noticed that a persons gender is often described wrongly.
%%
Similar uncertainty is also seen in telling apart various fruits and vegetables.
%%
This happens because these objects often occur in similar language contexts and hence
context cannot help differentiate them.
%%
To correctly identify them the LSTM generator needs to learn really fine grained
object classification.
%%

We hypothesize that this problem exacerbated by the fact that the language model
essentially needs to do a N-way classification at every time-step, where N is
the size of the vocabulary.
%%
When the size of the vocabulary is large, e.g.\@ on COCO dataset the vocabulary
contains 8790 words, this is a hard classification problem.
%%
Additional complexity arises because the errors made in this N-way
classification at any time-step affects all future predictions made during the
generation of that caption.

We attempt to address this problem by splitting the N-way classification into
two smaller hierarchical classification problems.
%%
To do this, the language model vocabulary is first split into $K$ classes and
each word in the vocabulary is assigned to one class. 
%%
Then the conditional probability of a word given the context can be factorized
into two parts as follows:
\begin{equation}
  \label{eq:class} 
  P(w_t | I,P, W^{t-1}) = P(c_t| I,P, W^{t-1})*P(w_t | c_t, I,P,W^{t-1})
\end{equation}
\noindent where $c_t$ is the class of word $w_t$ and $W^{t-1}$ is sequence of words
seen up to time $t-1$, ($w_0\cdots w_{t-1}$).

This allows us to split the N-way softmax at the LSTM output into two parts, a
$N_c$-way softmax to predict the correct class and a $N_{wc}$-way softmax to
predict the correct word within this class.
%%
Here $N_{wc}$ is the number of words within a class.
%%
This hierarchy allows the language model to learn separate decoders for each
class allowing it to possibly learn fine grained classification.
%%
The hierarchical structure is also more amenable to adding new words to the
vocabulary.
%%
Adding a new word to a specific class already transfers the knowledge the model
has about the class to the word, and the model only needs to learn the
distribution of the word within the class.

\subsection{Clustering words into classes}
The first step in implementing the hierarchical structure in the decoder is to
cluster words into different classes.
%%
There are two kinds of approaches to do this clustering in the literature.
%%
First group of methods use knowledge bases like WordNet to find similar words
and group them together.
%%
Second category of methods are completely data driven and use just the training
data statistics to build these clusters.
%%
Here we will look at two different methods relying solely on the COCO training
data to cluster words into classes.

\subsubsection*{Brown Clustering}
A popular data driven clustering algorithm is \emph{Brown clustering} proposed
in~\cite{BrownClust}.
%%
This is a hierarchical clustering algorithm which produces a hierarchical tree
like clusters of all the words in the vocabulary.
%%
To begin with, all the words in the vocabulary are assigned to individual leaf
nodes in a tree. 
%%
Then, starting from the leaf nodes, two nodes whose merging would reduce the
clustering cost function the most are combined into a single node.
%%
The clustering cost function is simply the perplexity assigned by a class based
bi-gram model to the training corpus. 
%%
In our case the training corpus is all the reference captions in the training
set.
%%
The clustering cost can be written as a function of the given cluster assignment 
, C, as below: 
\begin{equation}
  \label{eq:brown} 
        Quality(C) = \frac{1}{n} log \prod_{i=1}^{n} P(C(w_i)|C(w_{i-1})) P(w|C(w_i))
\end{equation}
\noindent where C, is the mapping which assigns words $w$ to their class $C(w)$

The process is terminated once the number of nodes have been reduced to the
desired number of clusters.
%%
Although this method produces hierarchical clusters, we ignore the hierarchy
and take all the classes in the final output as independent classes
in our language model 
\subsubsection*{K-means Clustering}

K-means is a simple yet very effective clustering algorithm widely used on
high-dimensional vectorial data.
%%
The algorithm tries to partition the set of input data vectors into $k$ clusters
such that each data point belongs to the cluster whose mean is closest to
it.
%%
Usually, euclidean distance is used to measure distance between data points.
%%
But we cannot directly use this method to cluster words as we cannot measure the
distance between words.

Instead, we can use the word-embeddings which are learnt in our language model
to represent words as vectors in d-dimensional space.
%%
With this, we can use the euclidean distance between word embeddings as a
measure of distance between the words and thus utilize K-means algorithm to
cluster words into $K$ classes.
%%
Once $K$ cluster centers are obtained, each word is assigned to its closest center
and this partitioning is used in the language model.

\subsection{Factorizing LSTM decoder output}
In order to implement the hierarchical decoder in our LSTM language model, we
need to split the decoder matrix, $W_d$ into a set of $K+1$ smaller matrices.
%%
This set contains one $W_{d\_cls}$ matrix of size $\text{LSTM hidden size}
\times k$ which is used to compute class probability.
%%
It also has K class specific matrices ${W_{d}^{1},W_{d}^{2}\cdots W_{d}^{K}}$,
each of the size $\text{LSTM hidden size}\times |c^k|$.
%%
Here $|c^k|$ is the size of class $c_k$.
%%
Consequently, computation performed to predict the word at time-step $t$ has two
stages, first to predict the right class of the word and then using the class
specific decoder matrix to predict the word within the class.
\begin{align}
        \label{eq:classLStmdecoder}
        P(c_{t}^{k}| I,P, W^{t-1}) &= \text{softmax}(W_{d\_cls} y(t)) \\
        c_t &= argmax_k\left(P(c_{t}^k| I,P, w^{t-1})\right) \\
        P(w_t | I,P, w^{t-1}) &= P(c_t| I,P, w^{t-1}) \cdot \text{softmax}(W_{d}^{c_t} y(t))
\end{align}

Although in theory this model should be faster since we only need to compute the
two smaller matrix multiplications instead of the one large one, due to
bottlenecks in our implementation, this hierarchical decoder runs slower than
the single stage decoder.
%%===========================================================================%%
\section{Ensembling Techniques}
%%----------------------------------%%
Using the many different image features and LSTM language model architectures we
have discussed before, we can train a set of different language models.
%%
When examining the pool of captions generated by such models for a set of
images, we have found out that different models tend to generate the best
captions for different kinds of images/video.
%%
This indicates that ensembling these different generative models could be a good
idea.
%%
If we could evaluate the suitability of a given caption for a
given image, we could possibly pick out the best candidate from the pool and
achieve better results than with any single model.
%%
In this section we will examine two methods of evaluating the suitability
of a caption to the input image/video and thus effectively ensembling multiple
language models.
%%

Concretely, given an input image/video, $V$, and a set of $p$ candidate
captions, $C_p = \left\{S_1,S_2,\cdots S_p \right\}$, generated by $m$ language
models, $LM = \left\{lm_1,lm_2,\cdots lm_m \right\}$, we wish to find the
evaluation function $E(S|V)$, such that $\text{argmax}_{S_i} E(S_i|V)$ is the
most suitable caption for $V$ in the set $C_p$.
%%
Note that this is different from using evaluation metrics to evaluate a
captioning system, as here we are assessing the captions without using the
ground truth. 
%%

First method relies on the caption generating language models themselves to
evaluate the candidates, while the second method involves training a separate
evaluator model to measure the similarity between the candidate caption and the
input.
%%
We will examine them in detail in the following subsections.

%%----------------------------------%%
\subsection{Combining Multiple models using Mutual Evaluation}

In this technique, we utilize the same language models which generated the
captions to also evaluate the captions.
%%
All our LSTM language models learn the conditional probability distribution of
the caption given the visual input, $P(S|V)$.
%%
We utilize this probability, $P(S|I)$, as a measure of the goodness of the
caption w.r.t the input, and hence use it to rank the candidate captions.
%%
Considering only the probability assigned to a caption $S_i$ by the language
model which generated it, we could pick the caption with the highest probability
score as the best caption.
%%
We refer to this method as "Self-Eval" in rest of the thesis. 


Alternatively, we could also get the candidate captions evaluated by all the
other models in the ensemble.
%%
Thus we get $m$ scores per caption, which can be used to pick the best candidate
either using the "max-mean" method or the "max-max" method.
%%
In the "max-mean" case, the candidate with the highest mean of probability
scores is chosen as the best caption.
%%
In case of the "max-max" method, the candidate with the highest maximum among
the $m$ probability scores is chosen as the best candidate.
\begin{eqnarray}
  \label{eq:cmme} 
  \text{max-mean:}& S_{best} &= \argmax_{S_j}\left(\frac{1}{m}\sum_{i=1}^{m}
  P_{lm_i}(S_j | V)\right)\\
  \text{max-max:}& S_{best} &= \argmax_{S_j} \left(\max_i P_{lm_i}(S_j |
  V)\right)
\end{eqnarray}
\noindent where $P_{lm_i}$ is the probability distribution learned by model
$lm_i$.

Since in this method models are evaluating each others sentences, we refer to it
as "Mutual-Eval". 
%%
This method is also similar to the peer-review model used in academic
publishing, where researchers with expertise in related fields evaluate each
others work.
%%
It works best when all the models in the ensemble are equally competitive, with
expertise in slightly different sub-domains of the dataset.

%%----------------------------------%%
\subsection{CNN Evaluator}
The language models we used in for evaluation of a candidate captions in the
previous subsection are trained generatively.
%%
Thus they need to learn to model both the semantic and syntactic structures of a
sentence and their relation to the input image.
%%
But, in our experiments, we have noticed that these language models are very
effective at learning to syntactic structures of sentences, and grammatical
mistakes are rare in the generated captions.
%%
Most of the errors in the generated captions are of semantic nature, wherein the
models tend to get the objects or the relations between the objects wrong.
%%
This could be because similar syntactic structures repeat across large number of
the reference captions in the training dataset, but similar semantic relations
are found only in smaller parts of the dataset.

This indicates that the evaluator function we use should mainly focus on
evaluating the semantic correctness of the candidate captions, and not worry
about nitty-gritties of syntactic correctness. 
%%
Thus, a generatively trained model will be suboptimal at this task and
discriminative training would more suitable. 
%%

Our solution to this is to discriminatively train a new model whose task is to
pick out the best candidate from the candidate set, given an input image or
video.
%%
We refer to this as an evaluator network.
%%
This network takes as input one visual feature vector and an input sentence and
computes a similarity score between the sentence and the visual feature. 
%%
The model is composed of a convolutional network to encode the sentences into a
sentence embedding, and a projection matrix which projects the visual feature
into the same space as the sentence embedding.
%%
A cosine similarity measure is used to evaluate the similarity between the
sentence embedding vector and the projected visual feature vector. 
%%

The convolutional network we use to encode sentences is based on the
paper~\cite{kim:2014:CNNsent}, where it is used for sentence sentiment
prediction.
%%
Figure~\ref{fig:CNNEval} shows the block diagram of our CNN-based evaluator.  
%%
Here, the input sentence is represented as a sequence of word vectors, which can
either be statically initialized with some standard word vector encodings, such
as GloVe~\cite{pennington2014glove} or word2vec~\cite{mikolov2013distributed},
or learned during the training phase.
%%
These word vectors are fed into a convolutional neural network which computes an
encoding of the sentence.

%%To compute the similarity between the image and the sentence encoding, the image
%%feature vector is first projected into same representation space using a
%%projection matrix.
%%%%
%%The similarity is then computed between the projected image vector and the
%%sentence encoding to give a match score between the input image and the
%%candidate sentence.

\begin{figure*}[t] 
  \centering
  \includegraphics[width=0.6\textwidth]{./images/CnnEval.pdf} 
  \caption{CNN based evaluator network to compute the similarity between 
    an image and a caption.}
  \label{fig:CNNEval} 
\end{figure*}

% Convolutional network on sentences

The first layer in the CNN consists of convolutional filters of different sizes.  
%%
All the filters operate over entire word vectors, but vary in the number of
words they cover, i.e., each filter is of the size $N_{\text{gram}} \times
N_{\text{word vec dim}}$. 
%%
We can have filters operating over bigrams, trigrams, etc.
%%
Specifically, our model has filters with bigrams, trigrams, 4-grams and 5-grams. 
%%
Additionally, we use multiple filters of any given size and so the total number
of convolutional filters in the evaluator model can be written as
$N_{\text{filt}} \cdot N_{\text{filt type}}$.

The filter outputs are max-pooled, which reduces the filter response over entire
sentence into a single scalar, i.e., its maximum response. 
%%
We can therefore think of each filter as looking for a specific n-gram,
disregarding its location within the sentence.
%%
These pooled outputs are concatenated and then projected to the desired vector
size to produce the final sentence encoding.

%% --------------------------------------

\subsubsection{Training the Evaluator Network}

The evaluator network needs to be trained to assign a high score for the best
caption and a lower scores for other captions.
%%
This is done by letting the network, for each training set image or video $V$,
to score its ground truth caption, $S^+$, and $k$ negative samples, $S_i^-,
i=1,\ldots,k$, drawn randomly from the ground truth captions of other media in
the training dataset.
%%
Now the training cost function $C$ is devised to maximize the score for the
positive sample and to minimize it for the negative samples. 
%%
This is achieved by applying a softmax on the scores of this batch (one positive
and $k$ negative samples) and maximize the softmax score of the positive sample:
%%
\begin{align}
  \label{eq:cnnprob} 
  P(S^+|S^-,I) &= \frac{\exp(-f(S^+,I))}{\exp(-f(S^+,I)) +
           \sum\limits_i^k{\exp(-f(S_i^- ,I))}} \\
  C &= -\log P(S^+|S^-,I) \;.
\end{align}
%%
Equation~(\ref{eq:cnnprob}) shows this computation with $f(S,I)$ representing
the similarity metric between the sentence candidate $S$ and media $V$.
%%
In our current method, we use the cosine similarity between the two vectors for
the purpose of $f(S,I)$.

%% --------------------------------------

\subsubsection{Utilizing the Evaluator Network}

Once trained, we can then use the evaluator network to compute the similarity
between each candidate in the pool of captions and the visual feature vector. 
%%
The candidate with the highest similarity is chosen as the output caption from
the ensemble for the input media.

%% ===========================================================================
