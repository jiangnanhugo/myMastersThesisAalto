\chapter{Introduction}
\label{chapter:intro}
** A picture is worth a thousand words, but how many can the machine say?**

The famous English adage states "A picture is worth a thousand
words"~\cite{ThousandQuote}. This is meant to convey that there is a lot the
viewer can learn/infer from a single still image and enumerating all the
information encoded in an image can use up even a thousand words. This is
illustrated in our extensive use of images in all forms of communication, from
scientific journals to Twitter chats. 
%
Humans are very good at processing images and videos and gathering all this
encoed information, but the computers still struggle to maks sense of the
simplest ones.
%
One could even say it is still easier for computers to store, parse, search and
even understand a thousand words than a single image.
%
The usage of multimedia on the internet has grown to staggering levels in the
last few yeas, due to easy access to cameras through smart phones. For eg. about
95 million photos are uploaded to Instagram every day~\cite{InstStats} and about
400 hours of video is uploaded to Youtube every minute~\cite{YouStats}. This
presents both an enoromous challenge and an oppurtunity to build smarter
computer algorithms to summarize and understand this data. Such algorithms could
hlep us index and search this huge amount of data better. \fixme{Something
something about general AI}.
%

One such problem at the heart of machine understanding of visual media, is
automatic captioning of images and videos. This involves designing an algoritm
which takes the image or the video as input and generates a natural language
caption describing succintly the content of the media. Effectively solving this
problem requires the machine to be able to identify the salient objects in the
image/video, their states and the relationship between these objects and also
correctly recognize the scene. It needs to be able to use this information to
generate a natural language caption summarizing it.

%%TODO: SHOULD THIS GO to literature review of features?
Until recently, the task of identifying even a single object in an image
reliably on a large datasets was hard. This changed dramatically with the
application of deep learning techniques, specifically convolutional neural
networks (CNN) for object recognition. On the basis of this success, deep
networks have been used succesfully to solve various tasks in computer vision.

In this thesis we will examine the task of automatic image/video captioning and
discuss algorithms based on deep learning to solve this task. We will discuss
the relavant literature, ....................................

Outlining the computer vision problem of labelling objects, how advent of CNN
has almost solved single object identification and to an extent multiple object
id and localization. With references ofcourse

More challenging problem is generating descriptive lables i.e captions
A precursor to the generic problem of machines understanding visual media. 

\fixme{(Piece together from the report and the ACMMM paper, about 1 page)}

\section{Problem statement}
Precisely describe the problem we are trying to solve here and all the
assumptions made in doing so.

Clearly outline the contributions and what the reader will get out of this
document. What are the different aspects of the problem we will address here,
how do we measure how well we are doing.
    - Basic Image captioning pipeline in detail
    - Then take apart each block and see how we can improve it
    - Then measure and demonstrate if these actually help
    - In depth analysis of  what are the strengths and weaknesses of current
    approaches
    - Put forth several ideas for what we could do next.

\fixme{Write Fresh 1-1.5 pages}

\section{Structure of the Thesis}
\label{section:structure} 
Describe thesis structure
\fixme{Write Fresh .5 pages}
