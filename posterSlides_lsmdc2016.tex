\documentclass{beamer}
\usetheme{Rochester}
\usecolortheme{beaver}
\setbeamertemplate{caption}{\raggedright\insertcaption\par}
\setlength{\itemsep}{\fill}

\usepackage{graphicx}
\usepackage{multirow}
\usepackage{longtable}

\def\cvs{${[}$Id: slides.tex,v 1.19 2015/12/09 16:42:43 shettyr1 Exp ${]}$}

\beamertemplatenavigationsymbolsempty
\bibliographystyle{acm}

\newcommand{\slide}[1]{\scalebox{1.0}{\includegraphics[page=#1]{slides}}}
\newcommand{\mct}[1]{\multicolumn{2}{c|}{\bf#1}}
\newcommand{\mcCell}[1]{\multicolumn{1}{|c|}{#1}}
\newcommand{\bs}{\small\bf}
\newcommand{\red}[1]{\protect\begin{color}{red}#1\protect\end{color}}
\newcommand{\green}[1]{\protect\begin{color}{green}#1\protect\end{color}}
\begin{document}

%%-----------------------------------------------------------------------------

\title{Frame- and Segment-Level Features and Candidate Pool Evaluation for Video Caption Generation}
\subtitle{ACM Multimedia Grand Challenge Presentation}
\author[Rakshith Shetty and Jorma Laaksonen]{Rakshith Shetty and Jorma Laaksonen}
\institute{Department of Computer Science, \\Aalto University}
\date{\today}

\frame{\titlepage} 

%%\frame{\frametitle{Table of contents}\tableofcontents} 

%%-----------------------------------------------------------------------------
\section{Overview}
\begin{frame}{Our Approach}
%\begin{itemize}
        %\item Standard Encoder--Decoder caption generation module.
        %\item Video featueres from two-different paradigms.
        %   \begin{itemize}
        %     \item Frame-level features for object and attributes
        %     \item Segment-level features for actions
        %   \end{itemize}
        %\item Train multiple caption generators with different feature combinations.
        %\item Ensemble using an evaluator network 
%\end{itemize} 
\begin{figure}[h]
    \centering
    \includegraphics[width=1.0\textwidth]{images/MMGC_Overview.pdf}
\begin{itemize}
    \item Popular encoder--decoder apporach
    \item An evaluator is used to ensemble multiple encoder-decoder caption generators 
    \item Trained only on MSR-VTT dataset. Does it generalize? 
\end{itemize}
\end{figure}
\end{frame}
%%-----------------------------------------------------------------------------
%\begin{frame}{Training and Generation}
%    \begin{itemize}
%        \item Pick a caption--image pair and maximize the probability assigned to caption 
%        \item Minimize the negative log likelihood 
%            \begin{align*}
%              \mathcal{NL}(w_0,\cdots, w_{L-1} | V) = -\sum_{t=0}^{L-1} \log(p(w_t|w_{t-1},V))
%            \end{align*}
%        \item Stochastic gradient descent using RMSProp 
%        \item Training limited to language model parameters 
%        \item Beamsearch is used in test mode to generate captions 
%    \end{itemize}
%\end{frame}
%%-----------------------------------------------------------------------------
%%=============================================================================
\section{Visual Features}
%%-----------------------------------------------------------------------------
\begin{frame}{Video Features}
    \centering
    \includegraphics[width=1.0\textwidth]{images/VideoFeatures.pdf}
\end{frame}
%%-----------------------------------------------------------------------------
\begin{frame}{Language Model}
    \begin{figure}[h]
        \centering
        \includegraphics[width=0.49\textwidth]{images/MultilayerResidualLSTM.pdf}
    \end{figure}
    \begin{itemize}
        \item Use multiple layers of LSTM with \textbf{Residual} connections
        \item Two input channels -- \emph{init} and \emph{persist} -- provide access to different types of features
    \end{itemize}
\end{frame}
%%-----------------------------------------------------------------------------
\section{Ensembling}
\begin{frame}{Why Ensembling?}
\begin{itemize}
        \item Dataset -- Various types of videos
        \item Features -- No single best feature
        \item Models -- Self-Evaluation can't be trusted
\end{itemize}
\underline{Example}
\begin{columns}       
    \column{0.4\linewidth}
    \begin{figure}[h]
        \centering
        \includegraphics[width=1.0\linewidth]{images/6774.png}
        \vfill
    \end{figure}
    \column{0.6\linewidth}
    \begin{figure}[thp]
      \begin{center}
      \centering
      \scalebox{0.9}{
      \begin{tabular}{|l|c|}
        \hline
        \textbf{\scriptsize\em Caption} &\textbf{\scriptsize\em Log-Prob}\\\hline
\textbf{\scriptsize\em \#1:} \scriptsize \red{a man is cooking}& \textbf{-3.89}  \\\hline
        \textbf{\scriptsize\em \#2:} \scriptsize \green{a man is eating food}& -5.20 \\\hline
        \textbf{\scriptsize\em \#3:} \scriptsize a man is cooking food & -6.64 \\\hline
      \end{tabular}
      }
        \vfill
      \end{center}
    \end{figure}
\end{columns}
\end{frame}
%%-----------------------------------------------------------------------------
\begin{frame}{Ensembling -- CNN Evaluator}
\begin{columns}       
    \column{0.4\linewidth}
    \begin{figure}[h]
        \centering
        \includegraphics[width=1.0\textwidth]{images/CnnEval.pdf}
        \vfill
    \end{figure}
    \column{0.6\linewidth}
    \begin{itemize}
        \item Evaluator trained discriminatively to compute similarity of visual input and caption. It consists of: 
           \begin{itemize}
               \item A CNN to encode the sentence 
               \item Visual feature embedding
               \item Cosine similarity of the two vectors 
           \end{itemize}
        \item Maximize the score assigned to correct caption--visual pair as opposed to negative pairs 
    \end{itemize}
\end{columns}
\end{frame}
%%-----------------------------------------------------------------------------
\begin{frame}{Does it work?}
\begin{columns}       
    \column{0.4\linewidth}
    \begin{figure}[h]
        \centering
        \includegraphics[width=1.0\linewidth]{images//6774.png}
        \vfill
    \end{figure}
    \column{0.6\linewidth}
    \begin{figure}[thp]
      \begin{center}
      \centering
      \scalebox{0.8}{
      \begin{tabular}{|l|c|c|}
        \hline
        \textbf{\scriptsize\em Caption} &\textbf{\scriptsize\em Log-Prob} &\textbf{\scriptsize\em CNN Score} \\\hline
\textbf{\scriptsize\em \#1:} \scriptsize a man is cooking& -3.89 & 0.86 \\\hline
        \textbf{\scriptsize\em \#2:} \scriptsize \green{a man is eating food}& -5.20 &\bf 0.89 \\\hline
        \textbf{\scriptsize\em \#3:} \scriptsize a man is cooking food & -6.64 & 0.87 \\\hline
      \end{tabular}
      }
        \vfill
      \end{center}
    \end{figure}
\end{columns}
\begin{table}[tbh]
  \centering
  \caption{Ensemble v/s single model on val set.}
  \scalebox{0.9}{
  \begin{tabular}{|l|c|c|c|c|c|}
    \hline
    Model &\bs BLEU-4 &\bs METEOR &\bs CIDEr &\bs ROUGE-L \\\hline\hline
    \small Best Single Model  & 0.409 & 0.268 & 0.433 & 0.598 \\
    \small CNN ensemble of 4 models & \bf0.411 & \bf0.277 & \bf0.464 & \bf0.596 \\\hline
  \end{tabular}
  }
  \label{tab:resVttFeat}
\end{table}
\end{frame}
%%-----------------------------------------------------------------------------
%%=============================================================================
%%-----------------------------------------------------------------------------
\begin{frame}{Adapting MSR-VTT Models for LSMDC}
    \begin{itemize}
        \item Same frame-level features are extracted from LSMDC videos
        \item The ground truth category information is replaced with the output of a category classifier
        \item C3D and trajectory features are also extracted from LSMDC
        \item Attempts to translate the output vocabulary to better match with LSMDC vocabulary were not frutiful
           \begin{itemize}
               \item Replacing common words like man, woman etc.\@ with SOMEONE
               \item Slightly better metrics but loses detail
           \end{itemize}
    \end{itemize}
\end{frame}
%%=============================================================================
\section{Conclusions}
%%-----------------------------------------------------------------------------
%\begin{frame}{Next.....}
%\tableofcontents[currentsection] 
%\end{frame}
%%-----------------------------------------------------------------------------
\begin{frame}{Conclusions}
\begin{itemize}
    \item MSR-VTT training is tempting due to large number of video--caption pairs (~ 130k)
    \item Finished last in the automatic metrics table 
       \begin{itemize}
           \item Largely due to huge mismatch in vocabulary and visual content
           \item Highlights drawbacks in the evaluation metrics as well
       \end{itemize}
    \item Produces better captions than the last year's winning model in several examples
    \item Fine-tuning or joint training might address these issues
\end{itemize}
\end{frame}
%%-----------------------------------------------------------------------------
%\begin{frame}{Sample Results on COCO}
%\begin{figure}[h]
%  \begin{center}
%  \centering
%  %\tabcolsep=0.05cm
%  \scalebox{0.7}{
%  \begin{tabular}{c|c|c|c|}
%          \cline{2-4}
%    &\includegraphics[width=0.25\linewidth,height=2.5cm]{images/COCO_val2014_000000502766.jpg} &
%    \includegraphics[width=0.25\linewidth,height=2.5cm]{images/COCO_val2014_000000161720.jpg} &
%    \includegraphics[width=0.25\linewidth,height=2.5cm]{images/COCO_val2014_000000385707.jpg} \\\hline
%    \mcCell{\textbf{\em\scriptsize C27:}}& \parbox[c][][c]{0.25\linewidth}{\smallskip \scriptsize a man and a dog herding sheep in a field\smallskip} &
%     \parbox[c][][c]{0.25\linewidth}{\smallskip \scriptsize a bathroom with a sink toilet and bathtub\smallskip} &
%     \parbox[c][][c]{0.25\linewidth}{\smallskip \scriptsize a bottle of wine and a glass of wine\smallskip}\\\hline
%     \mcCell{\textbf{\em\scriptsize C19:}}& \parbox[c][][c]{0.25\linewidth}{\smallskip \scriptsize a man standing next to a herd of sheep\smallskip}&
%     \parbox[c][][c]{0.25\linewidth}{\smallskip \scriptsize a bathroom with a toilet and a sink\smallskip}&
%     \parbox[c][][c]{0.25\linewidth}{\smallskip \scriptsize two bottles of wine sitting on a table\smallskip}\\\hline
%     &\includegraphics[width=0.25\linewidth,height=2.5cm]{images/COCO_val2014_000000251330.jpg} &
%    \includegraphics[width=0.25\linewidth,height=2.5cm]{images/COCO_val2014_000000218404.jpg} &
%    \includegraphics[width=0.25\linewidth,height=2.5cm]{images/COCO_val2014_000000119516.jpg} \\\hline
%    \mcCell{\textbf{\em\scriptsize C27:}}& \parbox[c][][c]{0.25\linewidth}{\smallskip \scriptsize a view of a bridge in the snow\smallskip} &
%     \parbox[c][][c]{0.25\linewidth}{\smallskip \scriptsize a table with plates of food on it\smallskip}&
%     \parbox[c][][c]{0.25\linewidth}{\smallskip \scriptsize a person riding a bike down a city street\smallskip}\\\hline
%     \mcCell{\textbf{\em\scriptsize C19:}}& \parbox[c][][c]{0.25\linewidth}{\smallskip \scriptsize a train crossing a bridge over a river\smallskip} &
%     \parbox[c][][c]{0.25\linewidth}{\smallskip \scriptsize a table topped with plates of food and drinks\smallskip}&
%     \parbox[c][][c]{0.25\linewidth}{\smallskip \scriptsize a city street filled with lots of traffic\smallskip}\\\hline
%  \end{tabular}
%  }
%  \end{center}
%  \caption{Captions generated for some images from the COCO validation set by two of our
%    models. The first row contains samples where the ensemble model,
%    C27, performs better, and the second row cases where C19 is
%    better.}
%  \label{fig:cococapSamps}
%\end{figure}
%
%\end{frame}
%%-----------------------------------------------------------------------------
\begin{frame}{Sample Captions Video}
\begin{figure}[thp]
  \begin{center}
  \centering
  \tabcolsep=0.10cm
  \scalebox{0.7}{
          \begin{tabular}{|l|l|l|}
    \hline
    \mcCell{\includegraphics[width=0.25\linewidth]{images/LsmdcVid0.png}} &
    \mcCell{\includegraphics[width=0.25\linewidth]{images/LsmdcVid1.png}} &
    \mcCell{\includegraphics[width=0.25\linewidth]{images/LsmdcVid2.png}}
    \\\hline
    \textbf{\scriptsize\em L6:} \scriptsize someone runs up to the car&
    \textbf{\scriptsize\em L6:} \scriptsize someone is dancing with her&
    \textbf{\scriptsize\em L6:} \scriptsize someone looks up at the house\\\hline\\\hline
    \mcCell{\includegraphics[width=0.25\linewidth]{images/9150.png}} &
    \mcCell{\includegraphics[width=0.25\linewidth]{images/9799.png}} &
    \mcCell{\includegraphics[width=0.25\linewidth]{images/7997.png}} \\\hline
    \textbf{\scriptsize\em M6:} \scriptsize a man is running in a gym &
    \textbf{\scriptsize\em M6:} \scriptsize a person is playing with a rubix cube &
    \textbf{\scriptsize\em M6:} \scriptsize cartoon characters are interacting\\
    \textbf{\scriptsize\em M2:} \scriptsize a man is running&
    \textbf{\scriptsize\em M2:} \scriptsize a man is holding a phone&
    \textbf{\scriptsize\em M2:} \scriptsize a person is playing a video game\\
    \textbf{\scriptsize\em M5:} \scriptsize a man is playing basketball&
    \textbf{\scriptsize\em M5:} \scriptsize a person is playing with a rubix cube &
    \textbf{\scriptsize\em M5:} \scriptsize a group of people are talking\\
    \hline
  \end{tabular}
  }
  \end{center}
  \label{fig:VttcapSamps}
\end{figure}
\end{frame}
%%-----------------------------------------------------------------------------
\begin{frame}{Qualitative Analysis}
\vspace{-5mm}
\begin{columns}
  \column{.5\linewidth}
  \begin{figure}[t]
    \centering
    \includegraphics[trim={2cm 0 2cm 0},clip,width=0.8\linewidth]{images/VTTCiderCateg.pdf}
    \vspace{-5mm}
    \caption{Performance by video category}%
  \end{figure}
  \column{.5\linewidth}
  \begin{figure}[t]
    \centering
    \includegraphics[trim={2cm 0 2cm 0},clip, width=0.8\linewidth]{images/VTTCiderLengths.pdf}
    \vspace{-5mm}
    \caption{Performance by video length}
    \label{fig:VttLenPerf}
  \end{figure}
\end{columns}
\begin{itemize}
    \item Video category has a huge impact
       \begin{itemize}
               \item \emph{How to}, \emph{video games} perform best 
               \item \emph{News}, \emph{tv} perform worst 
       \end{itemize}
    \item Surprisingly, length doesn't seem effect much (in-conclusive) 
\end{itemize}
\end{frame}
\begin{frame}[allowframebreaks]
        \frametitle{References}
        \bibliography{bib_thesis_used_only}
\end{frame}
%%-----------------------------------------------------------------------------
\begin{frame}{}
\begin{center}
    \Large Thank You for Listening\\[6mm]
    \Large Any Questions? 
\end{center}
\end{frame}
\end{document}
